%% start of file `template.tex'.
%% Copyright 2006-2015 Xavier Danaux (xdanaux@gmail.com), 2020-2021 moderncv maintainers (github.com/moderncv).
%
% This work may be distributed and/or modified under the
% conditions of the LaTeX Project Public License version 1.3c,
% available at http://www.latex-project.org/lppl/.


\documentclass[11pt,a4paper,sans]{moderncv}        % possible options include font size ('10pt', '11pt' and '12pt'), paper size ('a4paper', 'letterpaper', 'a5paper', 'legalpaper', 'executivepaper' and 'landscape') and font family ('sans' and 'roman')

% moderncv themes
\moderncvstyle{casual}                             % style options are 'casual' (default), 'classic', 'banking', 'oldstyle' and 'fancy'
\moderncvcolor{blue}                               % color options 'black', 'blue' (default), 'burgundy', 'green', 'grey', 'orange', 'purple' and 'red'
%\renewcommand{\familydefault}{\sfdefault}         % to set the default font; use '\sfdefault' for the default sans serif font, '\rmdefault' for the default roman one, or any tex font name
%\nopagenumbers{}                                  % uncomment to suppress automatic page numbering for CVs longer than one page

% character encoding
%\usepackage[utf8]{inputenc}                       % if you are not using xelatex ou lualatex, replace by the encoding you are using
%\usepackage{CJKutf8}                              % if you need to use CJK to typeset your resume in Chinese, Japanese or Korean

% adjust the page margins
\usepackage[scale=0.85]{geometry}
\setlength{\footskip}{136.00005pt}                 % depending on the amount of information in the footer, you need to change this value. comment this line out and set it to the size given in the warning
%\setlength{\hintscolumnwidth}{3cm}                % if you want to change the width of the column with the dates
%\setlength{\makecvheadnamewidth}{10cm}            % for the 'classic' style, if you want to force the width allocated to your name and avoid line breaks. be careful though, the length is normally calculated to avoid any overlap with your personal info; use this at your own typographical risks...

\usepackage[bibstyle=publist,plauthorhandling=highlight,backend=biber]{biblatex}
\plauthorname[Criston]{Hyett}
\bibliography{./publications.bib}                        % 'publications' is the name of a BibTeX file

% font loading
% for luatex and xetex, do not use inputenc and fontenc
% see https://tex.stackexchange.com/a/496643
\ifxetexorluatex
  \usepackage{fontspec}
  \usepackage{unicode-math}
  \defaultfontfeatures{Ligatures=TeX}
  \setmainfont{Latin Modern Roman}
  \setsansfont{Latin Modern Sans}
  \setmonofont{Latin Modern Mono}
  \setmathfont{Latin Modern Math} 
\else
  \usepackage[utf8]{inputenc}
  \usepackage[T1]{fontenc}
  \usepackage{lmodern}
\fi

% personal data
\name{Criston}{Hyett}
\address{2525 E Prince Rd, Apt 61}{Tucson, AZ, 85716}{}% optional, remove / comment the line if not wanted; the "postcode city" and "country" arguments can be omitted or provided empty
\phone[mobile]{+1~520.651.1433}                   % optional, remove / comment the line if not wanted; the optional "type" of the phone can be "mobile" (default), "fixed" or "fax"
\email{cmhyett@math.arizona.edu}                               % optional, remove / comment the line if not wanted
\homepage{cmhyett.github.io/home/}                         % optional, remove / comment the line if not wanted

% Social icons
\social[github]{cmhyett}                              % optional, remove / comment the line if not wanted
\social[googlescholar]{Criston Hyett}            % optional, remove / comment the line if not wanted

% bibliography adjustments (only useful if you make citations in your resume, or print a list of publications using BibTeX)
%   to show numerical labels in the bibliography (default is to show no labels)
%\makeatletter\renewcommand*{\bibliographyitemlabel}{\@biblabel{\arabic{enumiv}}}\makeatother
\renewcommand*{\bibliographyitemlabel}{[\arabic{enumiv}]}
\renewcommand{\listitemsymbol}{}
%   to redefine the bibliography heading string ("Publications")

% bibliography with mutiple entries
%\usepackage{multibib}
%\newcites{book,misc}{{Books},{Others}}
%----------------------------------------------------------------------------------
%            content
%----------------------------------------------------------------------------------
\begin{document}

%-----       letter       ---------------------------------------------------------
% recipient data
\recipient{New York University, Courant Institute of Mathematical Sciences}{Concerning: Postdoctoral Researcher - NYU Courant - Benjamin Peherstorfer's Group}
\date{\today}
\opening{To whom it may concern,}
\closing{I appreciate your consideration,}
%\enclosure[Attached]{curriculum vit\ae{}}          % use an optional argument to use a string other than "Enclosure", or redefine \enclname
\makelettertitle


%I am writing to apply as a postdoctoral researcher in Dr. Peherstorfer's group. I was particularly interested in the solicitation due to the stated focus on scientific machine learning and uncertainty quantification.n
I write to express my keen interest in the postdoctoral researcher position in Dr. Peherstorfer's group. I am particularly drawn to this opportunity due to the group's emphasis on advancing research in the mathematics of scientific machine learning and uncertainty quantification, as outlined in the solicitation. This specific focus resonates with my academic expertise, and aligns with my research interests and goals of professional growth.

%I am advised by Dr. Misha Chertkov working mainly on 2 projects:
In my current work as a PhD candidate in applied mathematics at the University of Arizona (UofA), advised by Dr. Misha Chertkov, I work mainly on 2 projects:
\begin{enumerate}
\item \textbf{Physics-informed machine learning (PIML) for Lagrangian turbulence.} In work from an interdisciplinary group from UofA and Los Alamos National Laboratory, we blend phenomenology and PIML to create reduced-order models for the statistical evolution of the Lagrangian velocity gradient tensor, with a focus on interpretability to inform turbulence theory.
  We presented our work at the American Physical Society Division of Fluid Dynamics conference, entitled ``Velocity gradient prediction using parameterized Lagrangian deformation models'' and plan to publish these findings in 2024.
  There are similarities between the approach we take and recent work from the group in neural Galerkin schemes and model reduction; I envision productive collaboration in this area.
  % we reduce ove from Navier-Stokes to a large number of nonlocal ODEs
% (We have not completed a paper here yet - but have one in its final stages following our APS talk of the same name ``Velocity gradient prediction using parameterized Lagrangian deformation models'' listed in the CV).
Further, while any view of turbulence must be inherently statistical, most applications of machine learning to the field seek to predict mean quantities - I see plenty of future work here via enriching predicted statistics, increasing the usability of trained ML models by quantifying/controlling errors, and more thoroughly leveraging the Lagrangian perspective.

\item \textbf{Simulation, analysis and control of natural gas networks.} We partner with the operations and planning division of NOGA Israel (Israel's independent system operator) to approach the problem of optimal gas flow (OGF) on a network using a physics-adherent approach. We published and presented our first work, largely analyzing networked gas-flow behavior under uncertainty via Monte Carlo, in Pipeline Simulation Interest Group's 2023 meeting. Our recently completed work focuses on OGF by implementing differentiable simulators to confine the optimization to the solution manifold of the gas-flow PDEs. We've submitted, and hope to present and publish this at Power Systems Computation Conference 2024. Finally, ongoing work looks to create a coupled natural gas/electric grid dynamic system for co-optimization. Integrating multi-fidelity modeling in both the physical and probabilistic spaces will be vital to mitigate computational complexity while creating a trustable product for our industry partners.
\end{enumerate}

These projects blend together dynamical systems, domain knowledge, and machine learning. This principled approach has allowed us to inform the phenomenology of turbulence and enrich the planning and operation of regional natural gas networks.

During the postdoc I look to continue application of these methodologies on scientific and engineering challenges, with the increased focus on the development of the methodologies themselves. To this end, I believe Dr. Peherstorfer's group would be a productive fit. We share the same vision for scientific machine learning, while approaching the interdisciplinary interface from different directions. Targets of my professional growth are the mathematical perspectives of efficient uncertainty quantification, and richer model-reduction strategies to inform physical theory. Finally, I hope to contribute and mentor through my knowledge of software development, computation, and physical insight.

\vspace{0.4in}
\makeletterclosing


\end{document}